%%%%%
%% Use \section{Title of appendix} to start a new appendix. Appendices are optional. 
%% If none are present, comment out "" in main.tex
%%%%%


\section{}\label{apx:appendix1}

Appendices should contain information that is too lengthy to be included in the thesis chapters but further support the conclusions of the thesis. For example, one could include additional experimental results in the form of tables or additional figures. Each appendix should start with a paragraph introducing the items being presented. Additionally, each table or figure should be preceded by a paragraph that explains the data being presented and the conclusions that can be drawn from them. Please note that data already presented in the main thesis should not be repeated in an appendix. One cannot, for example, include the results of an experiment as a figure in the thesis and again as a table in the appendix. One can, however, include detailed results in tables in the appendix and a summary figure in the thesis portraying only the ``best" results.

\subsection{Subsections in Appendices}\label{apx:appendix1:subsections}

It is possible, and advisable, to use (multiple levels of) subsections in an Appendix. Note, however, that the main Appendix sections are not titled. They will automatically be assigned Appendix A, Appendix B, etc. Sub-sections, however, can have titles and will be listed in the TOC as A.1, A.2, etc.

\section{}\label{apx:appendix2}

Different appendices should cover unrelated topics. If the appendices are on the same topic, they should just be sub-sections within the same appendix.